\documentclass[letterpaper,superscriptaddress,showkeys,longbibliography]{revtex4-1}\usepackage{graphicx, color}
%% maxwidth is the original width if it is less than linewidth
%% otherwise use linewidth (to make sure the graphics do not exceed the margin)
\makeatletter
\def\maxwidth{ %
  \ifdim\Gin@nat@width>\linewidth
    \linewidth
  \else
    \Gin@nat@width
  \fi
}
\makeatother

\IfFileExists{upquote.sty}{\usepackage{upquote}}{}
\definecolor{fgcolor}{rgb}{0.2, 0.2, 0.2}
\newcommand{\hlnumber}[1]{\textcolor[rgb]{0,0,0}{#1}}%
\newcommand{\hlfunctioncall}[1]{\textcolor[rgb]{0.501960784313725,0,0.329411764705882}{\textbf{#1}}}%
\newcommand{\hlstring}[1]{\textcolor[rgb]{0.6,0.6,1}{#1}}%
\newcommand{\hlkeyword}[1]{\textcolor[rgb]{0,0,0}{\textbf{#1}}}%
\newcommand{\hlargument}[1]{\textcolor[rgb]{0.690196078431373,0.250980392156863,0.0196078431372549}{#1}}%
\newcommand{\hlcomment}[1]{\textcolor[rgb]{0.180392156862745,0.6,0.341176470588235}{#1}}%
\newcommand{\hlroxygencomment}[1]{\textcolor[rgb]{0.43921568627451,0.47843137254902,0.701960784313725}{#1}}%
\newcommand{\hlformalargs}[1]{\textcolor[rgb]{0.690196078431373,0.250980392156863,0.0196078431372549}{#1}}%
\newcommand{\hleqformalargs}[1]{\textcolor[rgb]{0.690196078431373,0.250980392156863,0.0196078431372549}{#1}}%
\newcommand{\hlassignement}[1]{\textcolor[rgb]{0,0,0}{\textbf{#1}}}%
\newcommand{\hlpackage}[1]{\textcolor[rgb]{0.588235294117647,0.709803921568627,0.145098039215686}{#1}}%
\newcommand{\hlslot}[1]{\textit{#1}}%
\newcommand{\hlsymbol}[1]{\textcolor[rgb]{0,0,0}{#1}}%
\newcommand{\hlprompt}[1]{\textcolor[rgb]{0.2,0.2,0.2}{#1}}%

\usepackage{framed}
\makeatletter
\newenvironment{kframe}{%
 \def\at@end@of@kframe{}%
 \ifinner\ifhmode%
  \def\at@end@of@kframe{\end{minipage}}%
  \begin{minipage}{\columnwidth}%
 \fi\fi%
 \def\FrameCommand##1{\hskip\@totalleftmargin \hskip-\fboxsep
 \colorbox{shadecolor}{##1}\hskip-\fboxsep
     % There is no \\@totalrightmargin, so:
     \hskip-\linewidth \hskip-\@totalleftmargin \hskip\columnwidth}%
 \MakeFramed {\advance\hsize-\width
   \@totalleftmargin\z@ \linewidth\hsize
   \@setminipage}}%
 {\par\unskip\endMakeFramed%
 \at@end@of@kframe}
\makeatother

\definecolor{shadecolor}{rgb}{.97, .97, .97}
\definecolor{messagecolor}{rgb}{0, 0, 0}
\definecolor{warningcolor}{rgb}{1, 0, 1}
\definecolor{errorcolor}{rgb}{1, 0, 0}
\newenvironment{knitrout}{}{} % an empty environment to be redefined in TeX

\usepackage{alltt}
\usepackage[utf8]{inputenc}
\usepackage{color,dcolumn,graphicx,hyperref}
\usepackage{natbib}
\usepackage{footnote}
\usepackage{threeparttable}
\usepackage{scrextend}
\usepackage{setspace}
\usepackage[margin=1in]{geometry}

% stuff for coloring rows in tables - BEGIN
\usepackage{booktabs}% http://ctan.org/pkg/booktabs
\usepackage{colortbl}% http://ctan.org/pkg/colortbl
\usepackage{amsmath}% http://ctan.org/pkg/amsmath
\usepackage{xcolor}% http://ctan.org/pkg/xcolor
\usepackage{graphicx}% http://ctan.org/pkg/graphicx

\colorlet{tableheadcolor}{gray!25} % Table header colour = 25% gray
\newcommand{\headcol}{\rowcolor{tableheadcolor}} %
\colorlet{tablerowcolor}{gray!10} % Table row separator colour = 10% gray
\newcommand{\rowcol}{\rowcolor{tablerowcolor}} %
    % Command \topline consists of a (slightly modified) \toprule followed by a \heavyrule rule of colour tableheadcolor (hence, 2 separate rules)
\newcommand{\topline}{\arrayrulecolor{black}\specialrule{0.1em}{\abovetopsep}{0pt}%
            \arrayrulecolor{tableheadcolor}\specialrule{\belowrulesep}{0pt}{0pt}%
            \arrayrulecolor{black}}
    % Command \midline consists of 3 rules (top colour tableheadcolor, middle colour black, bottom colour white)
\newcommand{\midline}{\arrayrulecolor{tableheadcolor}\specialrule{\aboverulesep}{0pt}{0pt}%
            \arrayrulecolor{black}\specialrule{\lightrulewidth}{0pt}{0pt}%
            \arrayrulecolor{white}\specialrule{\belowrulesep}{0pt}{0pt}%
            \arrayrulecolor{black}}
    % Command \rowmidlinecw consists of 3 rules (top colour tablerowcolor, middle colour black, bottom colour white)
\newcommand{\rowmidlinecw}{\arrayrulecolor{tablerowcolor}\specialrule{\aboverulesep}{0pt}{0pt}%
            \arrayrulecolor{black}\specialrule{\lightrulewidth}{0pt}{0pt}%
            \arrayrulecolor{white}\specialrule{\belowrulesep}{0pt}{0pt}%
            \arrayrulecolor{black}}
    % Command \rowmidlinewc consists of 3 rules (top colour white, middle colour black, bottom colour tablerowcolor)
\newcommand{\rowmidlinewc}{\arrayrulecolor{white}\specialrule{\aboverulesep}{0pt}{0pt}%
            \arrayrulecolor{black}\specialrule{\lightrulewidth}{0pt}{0pt}%
            \arrayrulecolor{tablerowcolor}\specialrule{\belowrulesep}{0pt}{0pt}%
            \arrayrulecolor{black}}
    % Command \rowmidlinew consists of 1 white rule
\newcommand{\rowmidlinew}{\arrayrulecolor{white}\specialrule{\aboverulesep}{0pt}{0pt}%
            \arrayrulecolor{black}}
    % Command \rowmidlinec consists of 1 tablerowcolor rule
\newcommand{\rowmidlinec}{\arrayrulecolor{tablerowcolor}\specialrule{\aboverulesep}{0pt}{0pt}%
            \arrayrulecolor{black}}
    % Command \bottomline consists of 2 rules (top colour
\newcommand{\bottomline}{\arrayrulecolor{white}\specialrule{\aboverulesep}{0pt}{0pt}%
            \arrayrulecolor{black}\specialrule{\heavyrulewidth}{0pt}{\belowbottomsep}}%
\newcommand{\bottomlinec}{\arrayrulecolor{tablerowcolor}\specialrule{\aboverulesep}{0pt}{0pt}%
            \arrayrulecolor{black}\specialrule{\heavyrulewidth}{0pt}{\belowbottomsep}}%
% stuff for coloring rows in tables - END

\newcolumntype{L}{>{\arraybackslash}m{5cm}} % creates now column type to wrap text
\newcolumntype{B}{>{\arraybackslash}m{3cm}} % creates now column type to wrap text
\newcolumntype{C}{>{\arraybackslash}m{3.5cm}} % creates now column type to wrap text

\hypersetup
{
    colorlinks = true, linkcolor = blue, citecolor = blue, urlcolor = blue,
}

\begin{document}




\setstretch{2} % double space the entire manuscript

\title{Consuming altmetrics: some observations and lessons}

\author{Scott Chamberlain}
\email[E-mail: ]{scott@ropensci.org}
\affiliation{Biology Dept., Simon Fraser University, Burnaby, BC, Canada V5B 1E1}

\keywords{altmetrics; R; sotware; API}

\maketitle

\section*{Abstract}

XXXX

\section*{Introduction}

Altmetrics, or article level metrics, measure the impact of individual articles, or objects, usually at the object level, or the author level. This is in stark contrast to the impact factor, which is a proprietary summation of the impact of all articles in a journal (owned and calculated by Thomson Reuters ©). Altmetrics have many advantages over journal level metrics, including quantifying more than just citations, and provide metrics on a variety of impacts (e.g., discussed by the media (mentions in the news), discussed by the public (facebook likes, tweets), and importance to colleagues (citations)). 

Altmetrics can be consumed in a variety of contexts: as static text, images, or graphs alongside a pdf or website, as a javascript widget in a website and more. A use case that will, and should, be increasingly common is using scripting languages (e.g., Python, Ruby, R) to consume altmetrics on a computer locally (rather than in a browser), often for purposes of research on altmetrics themselves. Consuming altmetrics from this perspective is somewhat different than the use case in which a user looks at altmetrics in a web browser. 

This paper discusses altmetrics from the perspective of developing and using scripting interfaces for altmetrics. From this perspective, there are a number of considerations: where can you get altmetrics data; data standardization and consistency; putting altmetrics into context, or normalization; relavance of altmetrics; data provenance, and technical barriers.

\section*{Altmetrics data providers}

There are many publishers that are now presenting altmetrics alongside their papers on their websites. However, these publishers do not yet provide public facing APIs (Application Programming Interface) at the time of writing. There are four major entities that aggreagate and provide altmetrics data: PLoS, ImpactStory, and Altmetric, and Plum Analytics (see Table 1 for details). Plum Analytics does not have an open public facing interface or API, so will not be discussed further. There are a few other smaller scale altmetrics providers, such as CiteIn (\url{http://citedin.org/}) and ScienceCard (\url{http://sciencecard.org/}), but they will not be discussed futher here since they are relatively smaller. PLoS and ImpactStory have open APIs, while Altmetrics API limits API requests by hour and day, and by payed vs. non-payed accounts. PLoS provides data in JSON (JavaScript Object Notation) and XML (Extensible Metadata Language), ImpactStory in JSON only, and Altmetric in JSON and JSONP. In terms of granularity, PLoS provides much more granular data than the others, with daily, monthly and yearly totals; ImpactStory provides only total values; and Altmetric provides total values, plus incremental summaries of their proprietary Altmetric score. PLoS is a publisher, while the mission of the other two is to collect and provide altmetrics data. PLoS and ImpactStory are non-profit, while Altmetric is for-profit.

The three providers overlap in some sources of altmetrics they gather, but not all (see Appendix Table A1). The fact that there is some complementarity in sources opens the possibility that different metrics can be combined from across the different providers to get more a complete set of altmetrics. For those that are compelementary, this should be relatively easy, and we don't have to worry about data consistency. However, when they share data sources, data may not be consistent between providers for the same data source (see \emph{Data standardization and consistency} below).

One of the important aspects of altmetrics is that most of the data collected by altmetrics aggregators like ImpactStory is that they aren't creating the data themselves, but rather are collecting the data from other sources that have their own licences. Thus, data licenses for PLoS, ImpactStory, and Altmetric are generally restricted to match those of the original data provider (e.g., some data providers do not let anyone to cache data). 

Note that in discussing the three providers, we are only aware of the details of each provider that are open to the public. For example, Altmetric and Plum Analytics provide some or all of their services to paying customers, which we don't cover here. 

\begin{table}[!ht]
    \begin{threeparttable}[b]
    \caption{Details on the four largest altmetrics providers.}\label{tab:a} % title of Table
        \begin{tabular}{|l|l|l|l|l|}
            \hline
            Variable & PLoS & ImpactStory & Altmetric & Plum Analytics \\
            \hline
            Open API? & Yes & Yes & Limited\tnote{d} & No \\
            Data format & JSON,XML & JSON & JSON,JSONP & Unknown \\
            Granularity\tnote{b} & D,M,Y & T & I & Unknown \\
            API Authentication & None & API key & API key & Unknown \\
            Business type & Publisher & Altmetrics provider & Altmetrics provider & Altmetrics provider \\
            Business model & Non-profit & Non-profit & For-profit & For-profit \\
            Rate limiting & Not enforced & Not enforced\tnote{c} & 1 call/sec.\tnote{d} & Unknown \\
            Products covered & Articles & Many\tnote{e} & Articles & Many\tnote{f} \\
            Software clients & R\tnote{g} & R,Javascript\tnote{h} & R,Python\tnote{i} & Unknown \\
            \hline
        \end{tabular}
        \begin{tablenotes}
            \item[a] Payed accounts with perks
            \item[b] D: day; M: month; Y: year; T: total; I: incremental summaries
            \item[c] Note: They recommend delaying a few seconds between requests
            \item[d] Also hourly and daily limits enforced; using API key increases limits
            \item[e] articles, code, software, presentations, datasets
            \item[f] articles, code, software, presentations, datasets, books, theses, etc. (see \url{http://www.plumanalytics.com/metrics.html} for a full list)
            \item[g] R (\url{https://github.com/ropensci/alm})
            \item[h] R (\url{https://github.com/ropensci/rimpactstory}), Javascriopt (\url{https://github.com/highwire/opensource-js-ImpactStory})
            \item[i] R (\url{https://github.com/ropensci/rAltmetric}), Python (\url{https://github.com/lnielsen-cern/python-altmetric})
        \end{tablenotes}
    \end{threeparttable}
\end{table}

\section*{Data standardization and consistency}

Now that there are multiple providers for altmetrics data, data consistency is something to keep in mind. For example, PLoS, ImpactStory and Altmetric do collect altmetrics from some of the same data sources. Are the numbers they present to users the same for the same paper, or are they different due to different collection dates, data sources, or methods of collection? Each of the three providers of course has the right to collect metrics as needed for their purposes. However, as altmetrics consumers and researchers, we should have a clear understanding of the potential hazards when using altmetrics data for research or otherwise. 

I used a set of 600 DOIs for full articles from PLoS journals only - this way all three providers would have data on the papers. I collected metrics from each of the three providers for each of the 600 DOIs. However, only a subset of DOIs had data from all three providers, so the final sample size was 308. For each DOI I removed any missing values, and calculated the maximum difference between values (i.e., providers) and plotted the distribution of these maximum difference values for seven altmetrics that were shared among the providers (Fig. 1). Figure 1 shows that, at least with respect to absolute numbers, PMC metrics are very different among providers, while PLoS views (relavant only to PLoS ALM and ImpactStory) are somewhat less variable among providers. Most of the altmetrics are not very different among providers, with most values at zero, or no difference among providers. 







That was a rough overview of hundreds of DOIs. What do the differences among providers look like in more detail? I used a set of 20 DOIs from the 308 above, to show the value of each altmetric from each of the altmetrics providers for each of the 20 DOIs. Note that in some cases there is very close overlap in values for the same altmetric on the same DOI across providers, but in some cases the values are very different. 




\subsection*{A crosswalk among providers}

As discussed above, when similar data sources are collected by altmetrics providers, ideally, there would be a way to go between, for example, data from Twitter for PLoS, ImpactStory, and Altmetric. Each of the three providers of course has the right to collect metrics as needed for their purposes, but as altmetrics consumers, we should be able to compare data from the same source across providers. In Appendix Table A1, I provide a table to crosswalk metrics for the same data source among providers.

\section*{Data provenance}

Data for the same altmetrics resource could be calculated in different ways and collected at different times for the same object. The three providers already provide the date the metrics were updated. However, there is little information available, via their APIs at least, regarding how data were collected, and what, if any, calculations were done on the data before providing the data. The for-profit providers, Altmetric and Plum Analytics, especially have no obligation to share these, but the altmetrics community overall would benefit from transparency in how data are collected. 

A good step in the right direction is that ImpactStory provides a field named \emph{provenance\_url} with each metric data source. For example, for a recent paper \cite{piwowar2007}, a GET call to the ImpactStory API returns many metrics, one of which is 10 bookmarks on Delicious. Importantly, they also return the field \emph{provenance\_url}, in this case \url{http://www.delicious.com/url/9df9c6e819aa21a0e81ff8c6f4a52029}, which takes you directly to the human readble page on Delicous from where the data was collected. This is important for researchers as ideally all of our research is replicable. A nice bit about digital data such as altmetrics is that we can trace back final altmetrics from providers such as ImpactStory to their original source.

The PLoS ALM API provides something less obvious with respect to provenance, a field called \emph{events\_url}, which for the same paper above \cite{piwowar2007} returns 82 bookmarks on Citeulike, and the human readable link to where the data was collected \url{http://www.citeulike.org/doi/10.1371/journal.pone.0000308}. 

What is ideal with respect to data provenance? Is the link to where the original data was collected enough? Probably so, if no calculations were done on the original data before reaching users. However, some of the providers do give numbers which have been calculated. For example, ImpactStory puts some metrics into context by calculating some percentage relative to a reference set. Ideally, how this is done should be very clear, and replicable. 

%% -mention Altmetric history data && PLoS's history data...

\section*{Putting altmetrics in context, or Normalization}

Raw altmetrics data can be number of tweets, or number of html views on a publishers website. What do these numbers mean? How does the paper or dataset I care about compare to others? ImpactStory gives context to their scores by classifying scores along two dimensions: audience (scholars or public) and type of engagement (view, discuss, save, cite, recommend). Users can then determine whether a product (paper, dataset, etc.) was highly viewed, discussed, saved, cited, or recommended, and by scientists, or by the public. This abstracts away many details; however, users can drill down to the underlying data via their API and web interface.  Altmetric has a different approach. They provide context for only one metric, the altmetric score. This is a single aggregate metric, the calculation of which is not known. They do provide context for the altmetric score, including how it compares to a) all articles in the same journal, b) all articles in the same journal published within three weeks of the article, c) all articles in the Almetric database, and d) all articles in the Almetric database published within three weeks of the article. No context is given though for individual altmetrics (e.g., tweets). 

Future work should consider further dimensions of context. For example, XXXX

\section*{Relavance of altmetrics}

The internet has facilitated the existence of altmetrics, as measures of impact are all around us, and many are now machine readable. However, which altmetrics are relavant? More importantly, which altmetrics are relavant to the community you care about? XXXX. [NOT SURE IT MAKES SENSE TO KEEP THIS SECTION???]

\section*{Technical barriers to use}

Using altmetrics via a scripting language means that the user has to consider whether the data source is machine readable, how easy the data is to retrieve and manipulate once retrieved, and whether the user has to authenticate. First, of the three altmetrics providers discussed, all provide an API, which means their data is easily machine readable. Second, how data is returned to the user can be important if the user is interacting with the API directly. Fortunately, many libraries, or extensions, exist to a number of programming languages relavant to scholars, which deal with preparing the data for easy consumption for users (see Table 1 for links to libraries). Last, authentication can be a barrier to use in that if a user has to take extra step of authenticating, they may not bother. There are a variety of possible authentication methods, some of which include: a) no authentication, b) username and password pair, c) API key, and d) OAuth (including OAuth1 and OAuth2) (Table 1). These different options make sense in different use cases. The first, no authentication, used by PLoS, makes sense when an API is first released and testers are needed to get feedback. A benefit of an API with no authentication is the barrier to entry is lower. That is, if you don't have to ask a user to register to get an API key they are more likely to use the API. The second and third options, username/password pair and API key are relatively similar; API keys are used by both ImpactStory and Altmetric (Table 1). The last option, OAuth, is not used by any of the altmetrics providers. This authentication method is however used by many API providers. From the viewpoint of a consumer in a desktop scripting language, OAuth can be painful. What works better for scripting languages are the first three options.  

\section*{Conclusion}

XXXXXX

\section*{Acknowledgments}

I thank Martin Fenner for inviting me to write a paper in this special issue, and for feedback from Carl Boettiger, Karthik Ram, and X on earlier versions of this manuscript.
  
\section*{References}
\bibliography{refs}

\section*{Appendix A. Crosswalk table among providers.}

The following Table A1 provides a crosswalk between altmetrics data collected by the three data providers. Note that these variables relate to one another across providers, but the data may be collected differently, and so for example, altmetrics collected for Twitter may differ between PLoS, ImpactStory and Altmetric.

Here is an example of calling the API of each the three providers to combine data from different sources, for the DOI \emph{10.1371/journal.pone.0018657} \cite{piwowar2011} (Table A2). In this case, only altmetrics that are shared among the providers are presented. There are many metrics that have exactly the same values among providers, though there are differences, which could be explained by the difference in the date data was collected. For example, PLoS ALM gives 6719 for combined PLoS views, while ImpactStory gives 6497 views. This is likely explained by the fact that PLoS ALM data was last updated on May 16, 2013, while ImpactStory's data was last updated on April 24, 2013. There are some oddities, however. For example, Altmetric gives 68 tweets, while ImpactStory only gives 22 tweets. ImpactStory's data was updated more recently (April 24, 2013) than that of Altmetric (November 14, 2012), which suggests something different about the way tweets among the two providers are collected as updated date alone can not explain the difference. In fact, Table A1 shows that ImpactStory collects tweets from Topsy (\url{http://topsy.com/}), while Altmetric collects them in an undisclosed manner, which obviously leads to different results.




\begin{table}[!ht]
\begin{threeparttable}[b]
\caption{Data sources used in taxize, tasks available, and links to them}
\begin{tabular}{|B|B|L|C|}
\hline
\headcol Data source & PLoS\tnote{a} & ImpactStory\tnote{b} & Altmetric\tnote{c} \\
\hline
Biod & biod & No & No \\
\rowcol Connotea & connotea & No & No \\
General blogs & bloglines & No & No \\
\rowcol Nature blogs & nature & No & No \\
Postgenomic & postgenomic & No & No \\
\rowcol Researchblogging & researchblogging & No & No \\
WebOfScience citations & webofscience & No & No \\
\rowcol Dryad & No & dryad:total\_downloads package\_views & No \\
Figshare & No & figshare:views shares downloads & No \\
\rowcol Github & No & github:forks stars & No \\
PLoS Search & No & plossearch:mentions & No \\
\rowcol Slideshare & No & slideshare:favorites views comments downloads & No \\
Google+ & No & No & cited\_by\_gplus\_count \\
\rowcol MSM & No & No & cited\_by\_msm\_count \\
News articles & No & No & Yes \\
\rowcol Reddit & No & No & cited\_by\_rdts\_count \\
Citeulike & citeulike & citeulike:bookmarks & No \\
\rowcol Crossref & crossref & plosalm:crossref\tnote{d} & No \\
PLoS ALM & counter(pdf\_views + html\_views)\tnote{e} & plosalm(html\_views, pdf\_views) & No \\
\rowcol PMC & pmc & plosalm:pmc\_full-text + pmc\_pdf\tnote{f} & No \\
PubMed & pubmed & pubmed:pmc\_citations\tnote{g} & No \\
\rowcol Scienceseeker & scienceseeker & scienceseeker:blog\_posts & No \\
Scopus citations & scopus & plosalm:scopus\tnote{h} & No \\
\rowcol Wikipedia & wikipedia & wikipedia:mentions & No \\
Delicious & No & delicious:bookmarks & cited\_by\_delicious\_count \\
\rowcol Facebook & facebook & facebook:shares\tnote{i} & cited\_by\_fbwalls\_count \\
Mendeley readers & mendeley shares & mendeley readers\tnote{j} & mendeley readers \\
\rowcol Twitter & twitter & topsy:tweets\tnote{k} & cited\_by\_tweeters\_count \\
\hline
\end{tabular}
\begin{tablenotes}
    \item[a] These are the exact names for each data source in the PLos ALM API. For example: \url{http://alm.plos.org/api/v3/articles?ids=10.1371/journal.pone.0018657&source=twitter}.
    \item[b] You can not request a specific source from the ImpactStory API, so these are the names of the fields in the returned json. For example, see the json from this call: \url{http://api.impactstory.org/v1/item/doi/10.1371/journal.pone.0018657?key=YOURAPIKEY}.
    \item[c] You can not request a specific source from the Altmetric API, so these are the names of the fields in the returned json. For example, see the json from this call: \url{http://api.altmetric.com/v1/doi/10.1371/journal.pbio.0018657?key=YOURAPIKEY}.
    \item[d] Collected from the PLoS ALM API. 
    \item[e] PLoS ALM also provides xml\_views. 
    \item[f] Collected from the PLoS ALM API. Other PMC data fields collected from PLoS ALM (pmc\_abstract, pmc\_supp-data, pmc\_figure, pmc\_unique-ip) and from PubMed (suppdata\_views, figure\_views, unique\_ip\_views, pdf\_downloads, abstract\_views, fulltext\_views).
    \item[g] Should be equivalent to plosalm:pubmed\_central. ImpactStory also collects pubmed:pmc\_citations\_reviews f1000 pmc\_citations\_editorials.
    \item[h] Collected from the PLoS ALM API. Scopus citations also collected from Scopus itself, in the field scopus:citations.
    \item[i] ImpactStory also collects Facebook clicks, comments, and likes. 
    \item[j] ImpactStory also collects Mendeley readers by discipline, number of groups that have added the article, percent of readers by country, and percent of readers by career\_stage. 
    \item[k] ImpactStory also collects the number of influential\_tweets from Topsy.
\end{tablenotes}
\end{threeparttable}
\end{table}

\end{document}
